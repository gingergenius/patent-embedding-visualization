%% LaTeX2e class for student theses
%% sections/abstract_de.tex
%% 
%% Karlsruhe Institute of Technology
%% Institute for Program Structures and Data Organization
%% Chair for Software Design and Quality (SDQ)
%%
%% Dr.-Ing. Erik Burger
%% burger@kit.edu
%%
%% Version 1.3.3, 2018-04-17

\Abstract
Die Menge der Veröffentlichungen, die den wissenschaftlichen Fortschritt dokumentieren, wächst kontinuierlich.
Dies erfordert die Entwicklung der technologischen Hilfsmittel für eine effiziente Analyse dieser Werke.
Solche Dokumente kennzeichnen sich nicht nur durch ihren textuellen Inhalt, sondern auch durch eine Menge von Metadaten-Attributen verschiedenster Art, unter anderem Beziehungen zwischen den Dokumenten.
Diese Komplexität macht die Entwicklung eines Visualisierungsansatzes, der eine Untersuchung der schriftlichen Werke unterstützt, zu einer notwendigen und anspruchsvollen Aufgabe.
Patente sind beispielhaft für das beschriebene Problem, weil sie in großen Mengen von Firmen untersucht werden, die sich Wettbewerbsvorteile verschaffen oder eigene Forschung und Entwicklung steuern wollen.

Vorgeschlagen wird ein Ansatz für eine explorative Visualisierung, der auf Metadaten und semantischen Embeddings von Patentinhalten basiert ist.
Wortembeddings aus einem vortrainierten Word2vec-Modell werden genutzt, um Ähnlichkeiten zwischen Dokumenten zu bestimmen.
Darüber hinaus helfen hierarchische Clusteringmethoden dabei, mehrere semantische Detaillierungsgrade durch extrahierte relevante Stichworte anzubieten.
Derzeit dürfte der vorliegende Visualisierungsansatz der erste sein, der semantische Embeddings mit einem hierarchischen Clustering verbindet und dabei diverse Interaktionstypen basierend auf Metadaten-Attributen unterstützt.

Der vorgestellte Ansatz nimmt Nutzerinteraktionstechniken wie Brushing and Linking, Focus plus Kontext, Details-on-Demand und Semantic Zoom in Anspruch.
Dadurch wird ermöglicht, Zusammenhänge zu entdecken, die aus dem Zusammenspiel von 1) Verteilungen der Metadatenwerten und 2) Positionen im semantischen Raum entstehen.

Das Visualisierungskonzept wurde durch Benutzerinterviews geprägt und durch eine Think-Aloud-Studie mit Patentenexperten evaluiert.
Während der Evaluation wurde der vorgestellte Ansatz mit einem Baseline-Ansatz verglichen, der auf \gls{tf-idf}-Vektoren basiert.
Die Benutzbarkeitsstudie ergab, dass die Visualisierungsmetaphern und die Interaktionstechniken angemessen gewählt wurden.
Darüber hinaus zeigte sie, dass die Benutzerschnittstelle eine deutlich größere Rolle bei den Eindrücken der Probanden gespielt hat als die Art und Weise, wie die Patente platziert und geclustert waren.
Tatsächlich haben beide Ansätze sehr ähnliche extrahierte Clusterstichworte ergeben.
Dennoch wurden bei dem semantischen Ansatz die Cluster intuitiver platziert und deutlicher abgetrennt.

Das vorgeschlagene Visualisierungslayout sowie die Interaktionstechniken und semantischen Methoden können auch auf andere Arten von schriftlichen Werken erweitert werden, z. B. auf wissenschaftliche Publikationen.
Andere Embeddingmethoden wie Paragraph2vec \cite{Le2014} oder BERT \cite{Devlin2018} können zudem verwendet werden, um kontextuelle Abhängigkeiten im Text über die Wortebene hinaus auszunutzen.
