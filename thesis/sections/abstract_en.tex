%% LaTeX2e class for student theses
%% sections/abstract_en.tex
%% 
%% Karlsruhe Institute of Technology
%% Institute for Program Structures and Data Organization
%% Chair for Software Design and Quality (SDQ)
%%
%% Dr.-Ing. Erik Burger
%% burger@kit.edu
%%
%% Version 1.3.3, 2018-04-17

\Abstract
The number of written works describing scientific progress is steadily increasing, which necessitates development of supportive tools for their efficient analysis.
These documents are characterized not only by their textual content, but also by a number of metadata attributes of various kinds, including any relationships between documents.
This complexity makes development of a visualization approach to aid examination of written works a challenging task.
Patents exemplify this problem as large amounts of them are studied by companies to gain competitive advantages and guide research and development efforts.

We propose an approach for explorative visualization based on both metadata and semantic embeddings of patent’s content.
Word embeddings from a pre-trained word2vec model are used to determine similarities between documents. 
Moreover, hierarchical clustering methods help provide several levels of semantic detail with via extracted relevant key terms.
To the best of our knowledge, no existing visualization approach combines semantic embeddings with hierarchical clustering while supporting various interaction types based on metadata attributes.

Our approach makes use of user interaction techniques such as brushing and linking, focus plus context, details on demand and semantic zoom. 
Because of that, it becomes possible to examine the patterns that result from the interplay between 1) distributions of metadata values and 2) positions in the semantic space.

The visualization concept is shaped by user interviews and evaluated via a think-aloud study with patent experts. 
During the evaluation we compared our approach to a baseline approach based on \gls{tf-idf} vectors. 
The usability study indicated that visualization metaphors and interaction techniques were appropriately chosen.
Moreover, it showed that the user interface of the prototype played a much larger role in participants' impression than the way patents are situated and clustered.
In fact, both approaches resulted in very similar extracted cluster key terms.
Nevertheless, the semantic approach resulted in more intuitive relative placement of clusters and better separation of clusters.

Proposed visualization layout, interaction techniques and semantic methods may be extended to other kinds of text documents, i. e. scientific publications. 
Other embedding methods such as paragraph2vec \cite{Le2014} or BERT \cite{Devlin2018} could be used to take advantage of contextual dependencies in text above the level of single words.