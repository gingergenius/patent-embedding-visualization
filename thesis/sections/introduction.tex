%% LaTeX2e class for student theses
%% sections/content.tex
%% 
%% Karlsruhe Institute of Technology
%% Institute for Program Structures and Data Organization
%% Chair for Software Design and Quality (SDQ)
%%
%% Dr.-Ing. Erik Burger
%% burger@kit.edu
%%
%% Version 1.3.3, 2018-04-17

\chapter{Introduction}
\label{ch:Introduction}

%To gain competitive advantages and guide research and development efforts, companies analyze patents in a given domain, with a patent landscape as a result. 
%Creating such a landscape is a cognitively complex task considering the amount of text and metadata (registration country, assignee company, patent classification, etc.) in each patent document and the large number of documents.

\section{Background and motivation}

Semantic embeddings are used in \gls{nlp} to capture relationships between text documents. 
However, positions and distances in the embedding space are not easily explainable and can hardly be understood by a user by themselves. 
Additional data dimensions incorporated into the representation of a semantic space provide immense added value. 
This is especially the case when visually exploring large document collections, where human perception must be aided in the task of finding patterns in data to prevent cognitive overload.

One example of a task in which such exploration takes place is \textit{patent landscaping}.
It ``constitutes an overview of patenting activity in a field of technology [...] and seeks to present complex information about this activity in a clear and accessible manner'' \cite{Trippe2015}.
Patents are an enormously valuable source of technology intelligence. 
They exemplify the problem at hand because they are text documents with a clearly defined structure, lots of metadata and references to other patent documents. 
With help of patent landscaping, companies acquire competitive advantages and steer their research and development efforts.

With about 3.1 million patent applications filed worldwide in 2016 \cite{WorldIntellectualPropertyOrganizationWIPO2017} and thousands of patent documents subject to analysis for a single domain, an effective approach facilitating the analysis is crucial.

\section{Objective and research questions}
\label{sec:objectives}

%explorative data visualization which incorporates semantic embeddings and faceted metadata.

The objective of this work is to provide a solution for the problem of exploration of large document collections.
The proposed visualization approach should take particularities of the patent domain into account and therefore be an efficient aid in the task of patent landscaping.
At the same time, the proposed approach should be generalizable for application on various kinds of text documents.

The objective presents a number of challenges that have to be addressed. 
They are described in \autoref{sec:challenges}.

Research questions that are being asked in this thesis are: 
\begin{itemize}
\item How can semantic embeddings be displayed in an transparent and explainable way?
\item How can semantic information enhance visual exploration of large document collections?
\item How can metadata of various types be combined with the semantic dimension through user interaction?
\item Do semantic embeddings provide added value compared with traditional frequency-based representations?
\end{itemize}

\section{Challenges}
\label{sec:challenges}

\subsection{Characteristics of data}
\subsubsection{Vocabulary}

Language and especially vocabulary in patent documents deviate significantly from generic written language.
Essentially, patents are written in a very abstract way, so that they protect a higher number of potential embodiments (see \autoref{subsec:participant_alpha} for details).
This complicates searching for similarities and differences in patent texts.

Whenever the vocabulary used in patents is not too general, it is likely to be very specific.
Technical terminology used to describe inventions is very unlikely to be contained in generic text corpora.
Moreover, frequency-based text processing methods are susceptible to inaccuracies when a large part of vocabulary consists of rare technical terms.
This is why a fine-tuning of the algorithm parameters related to term frequency is necessary.
We address this 1) by using a model trained specifically on patent vocabulary (see \autoref{subsec:data_source})  and 2) by carefully adjusting parameters for the key term extraction algorithm, especially for cluster key terms as described in \autoref{subsec:hierarchical_clustering}.

\subsubsection{Dimensionality and data types}

The documents we are dealing with consist of textual parts and metadata. 
High-dimensional text representations are necessary to represent content of patents.
It is a challenge to map them onto a lower-dimensional visualization space in a beneficial way.
Additionally, the visualization approach we aim to develop has to combine semantics gained from text with various visual dimensions derived from metadata of different types.
We experiment with multiple dimension reduction techniques as described in \autoref{subsec:embeddings}.

\subsubsection{Visual scalability}

The datasets that are being analyzed by patent experts can consist of hundreds to thousands of documents.
This is not ``big data'' in the classic sense of the word, but it definitely is on the upper end of the spectrum when it comes to visual representation.
The challenge is to develop an approach that works equally well for a wide range of dataset sizes.
It should be able to display thousands of documents in a comprehensive way.
For that, the proposed visualization approach has to use screen space wisely and leverage different levels of detail to avoid overwhelming the user.
This is why we provide different levels of detail via semantic zooming as described in \autoref{sec:outline_visualization_concept}.

\subsection{Evaluation}

The patent domain has been named in \cite{Chen2005} as an area where visualization has potential high-impact as a medium for finding causality, forming hypotheses and assessing available evidence.
This makes interactive visualization an attractive research topic. 
At the same time, the nature of the cognitive processes involved makes evaluation difficult.

\cite{Carpendale2008} argues that a great variety of cognitive reasoning tasks exists.
Low-level detailed tasks such as compare, contrast, cluster are more clearly defined.
High-level complex cognitive tasks include understanding of data trends, uncertainties, causal relationships or learning a domain.
No clear definition exists for some of those tasks, so they are challenging to test empirically.

When testing visualization approaches with experts, success in a task may be attributed to an interplay between expert's 1) meta-knowledge, 2) knowledge from other sources and 3) knowledge gained from the presented data.
This complicates interpretation of evaluation results further.
We address this in our evaluation by having multiple tasks per hypothesis that we evaluate as described in \autoref{ch:evaluation}.

\section{Structure of the thesis}
\label{sec:structure_of_thesis}

After the introduction in the first chapter, in \autoref{ch:related_work} we establish some fundamental concepts our approach builds upon.
Among other things, we introduce some definitions from the patent domain that are used throughout this thesis.
We then review the state of the art for data visualization approaches that provide means to explore scientific publications or patents.

Then, in \autoref{ch:case_study} we define the framework for the case study to be conducted.
The methodology is being established: the frame of reference of the study, the methods for data collection, etc.
We justify our choice of semi-structured user interviews for the formative part and think-aloud study followed by a \gls{sus} questionnaire for the summative part.
We then describe the course of the discussion during interviews with patent experts and summarize our findings and their effect on the development of a visualization concept.

In \autoref{ch:concept}, we first briefly outline the concept for the visualization.
We propose a visualization layout consisting of connected views of different kinds of interactive charts: scatter plot, histogram, sunburst with breadcrumbs and detail view.
which implement principles of focus + context, brushing and linking, semantic zoom and Shneiderman's information visualization mantra
Next, we justify the choice of a two-dimensional visualization space.
We then describe how the initial idea appeared and how the concept evolved.
Finally, we briefly cover how we derive semantic representations of documents, cluster them and extract relevant key terms to provide interpretability.

Next, \autoref{ch:implementation} goes into detail about the data processing necessary to prepare patent documents for visualization: preprocessing and cleaning, computing document vectors, dimension reduction, extracting relevant key terms, hierarchical clustering, etc.
It also contains a detailed description of the elements in the user interface of the visualization as they were implemented in the prototype.
We then elaborate on the interactions between the coordinated views of the prototype.

\autoref{ch:evaluation} establishes hypotheses about the visualization approach that pertain to the different visual elements and aspects of interaction.
Then, the hypotheses are evaluated through a think-aloud study, which detects usability problems and verifies how suitable the developed approach is for supporting exploration.
We confirm that the visualization metaphors and interaction techniques were chosen appropriately.
Moreover, the study shows that the user interface of the prototype played a much larger role in participants' impression than the way patents are situated and clustered.

Finally, \autoref{ch:conclusion} summarizes the proposed approach and the key findings of this work.
We conclude by discussing possible improvements both in a general sense and specifically related to patent domain.